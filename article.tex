%-----------------------------------------------------------------------------
%
%               Template for sigplanconf LaTeX Class
%
% Name:         sigplanconf-template.tex
%
% Purpose:      A template for sigplanconf.cls, which is a LaTeX 2e class
%               file for SIGPLAN conference proceedings.
%
% Guide:        Refer to "Author's Guide to the ACM SIGPLAN Class,"
%               sigplanconf-guide.pdf
%
% Author:       Paul C. Anagnostopoulos
%               Windfall Software
%               978 371-2316
%               paul@windfall.com
%
% Created:      15 February 2005
%
%-----------------------------------------------------------------------------


\documentclass[9pt]{sigplanconf}

% The following \documentclass options may be useful:
%
% 10pt          To set in 10-point type instead of 9-point.
% 11pt          To set in 11-point type instead of 9-point.
% authoryear    To obtain author/year citation style instead of numeric.

\usepackage{amsmath}

\begin{document}

\conferenceinfo{FOOL '12}{October 22, 2012, Tucson, AZ, USA.} 
\copyrightyear{2012} 
\copyrightdata{[to be supplied]} 

\title{Dependent Object Types}
\subtitle{A foundations for Scala's type system} % TODO: ask

\authorinfo{Nada Amin \and Adriaan Moors \and Martin Odersky}
           {EPFL}
           {first.last@epfl.ch}

\maketitle

\begin{abstract}
We propose a new type-theoretic foundation of Scala and languages like
it: the Dependent Object Types calculus (DOT). DOT models Scala's
path-dependent types and abstract type members, as well as its mixture
of nominal and structural typing through the use of refinement
types. It makes no attempt to model inheritance or mixing
composition. The calculus does not model what's currently in Scala: it
is more normative than descriptive.

We show that DOT and its patched-up variants are not syntactically
sound, by exhibiting counterexamples to preservation. Nevertheless, we
prove type-safety of the calculus via step-indexed logical relations.
\end{abstract}

\category{D.3.3}{Language Constructs and Features}{Abstract data types, Classes and objects, polymorphism}
\category{D.3.1}{Formal Definitions and Theory}{Syntax, Semantics}
\category{F.3.1}{Specifying and Verifying and Reasoning about Programs}{}
\category{F.3.3}{Studies of Program Constructs}{Object-oriented constructs, type structure}
\category{F.3.2}{Semantics or Programming Languages}{Operational semantics}

\terms
Languages, Theory, Verification

\keywords
calculus, objects, dependent types, step-indexed logical relations

%\section{Introduction}
%\appendix
%\section{Appendix Title}
%\acks

\bibliographystyle{abbrvnat}
\bibliography{dot}

\end{document}
