\documentclass{beamer}


% The following \documentclass options may be useful:
%
% 10pt          To set in 10-point type instead of 9-point.
% 11pt          To set in 11-point type instead of 9-point.
% authoryear    To obtain author/year citation style instead of numeric.

\usepackage{amsmath}
\usepackage{amssymb}

\usepackage{fleqn}
\usepackage{listings}
\usepackage{math}
\usepackage{amsmath}
\usepackage{latexsym}
\usepackage{bcprules}
\usepackage[scaled=0.848971]{luximono} % This is for 11 pt Default font
\usepackage[T1]{fontenc}

% Prooftree formatting
\usepackage{prooftree}

\usepackage{multicol}
\usepackage{framed}

\usepackage{stmaryrd}

%\usepackage{float}
%\floatstyle{boxed} 
%\restylefloat{figure}

% support for generating PDF files
%\newif\ifpdf
%    \ifx\pdfoutput\undefined
%    \pdffalse
%\else
%    \pdftrue
%    \pdfoutput=1
%\fi

%versions
% Use dependent function types
\newif\ifdep\depfalse

\lstset{
  literate=
  {=>}{$\Rightarrow\;$}{2}
  {<:}{$<:\;$}{1}
}

\lstdefinelanguage{scala}{% 
       morekeywords={% 
                try, catch, throw, private, public, protected, import, package, implicit, final, package, trait, type, class, val, def, var, if, for, this, else, extends, with, while, new, abstract, object, case, match, sealed,override},
         sensitive=t, % 
   morecomment=[s]{/*}{*/},morecomment=[l]{\//},% 
   mathescape,
%   escapeinside={/*\%}{*/},%
   rangeprefix= /*< ,rangesuffix= >*/,%
   morestring=[d]{"}% 
 }
 
\lstset{breaklines=true,language=scala} 

\def\code{\lstinline}  % shorter version so you can write \code|String[Foo]|
                       % -- \def must be in same file as uses for this to
                       % work...
\newcommand{\lstref}[1]{Listing~\ref{#1}}
\newcommand{\Lstref}[1]{Listing~\ref{#1}} % only capitalise at beginning of sentence?
\newcommand{\secref}[1]{Section~\ref{#1}}
\newcommand{\Secref}[1]{Section~\ref{#1}} % only capitalise at beginning of sentence?



% \lstset{basicstyle=\footnotesize\ttfamily, breaklines=true, language=scala, tabsize=2, columns=fixed, mathescape=false,includerangemarker=false}
% thank you, Burak 
% (lstset tweaking stolen from
% http://lampsvn.epfl.ch/svn-repos/scala/scala/branches/typestate/docs/tstate-report/datasway.tex)
\lstset{
    xleftmargin=2em,%
    framesep=5pt,%
    frame=none,%
    captionpos=b,%
    fontadjust=true,%
    columns=[c]fixed,%
    keepspaces=false,%
    basewidth={0.56em, 0.52em},%
    tabsize=2,%
    basicstyle=\small\tt,% \small\tt
    commentstyle=\textit,%
    keywordstyle=\bfseries,%
    escapechar=\%,%
}

%% set latex/pdflatex specific stuff
%\ifpdf
    \usepackage[pdftex,
                hyperindex,
                plainpages=false,
                breaklinks,
                colorlinks,
                citecolor=black,
                filecolor=black,
                linkcolor=black,
                pagecolor=black,
                urlcolor=black]{hyperref}
    \usepackage[pdftex]{graphicx}
    \DeclareGraphicsExtensions{.jpg,.pdf}
    \pdfcatalog {
        /PageMode (/UseNone)
    }
    \usepackage{thumbpdf}
    \usepackage[pdftex]{color}
%\else
%    \usepackage[ps2pdf]{hyperref}
%    \usepackage{graphicx}
%    \DeclareGraphicsExtensions{.eps,.jpg}
%    \usepackage{color}
%\fi

%\setlength{\parindent}{0pt}
%\setlength{\parskip}{5pt}

% verbfilter stuff
\newcommand{\prog}[1]{{\sl #1}}
\newenvironment{program}[1][10.5]
  {\fontsize{#1}{13.6}\tt\begin{tabbing}\hspace*{0.5\parindent}\=\+\kill}
  {\end{tabbing}\noindent}
\newcommand{\blockcomment}[1]{{\color{grayPoint3}#1}}
\newcommand{\linecomment}{\color{grayPoint3}}
\newcommand{\grey}{\color{grey}}

%\newenvironment{program}{\ \ \ \ \begin{minipage}{\textwidth}\renewcommand{\baselinestretch}{1.0}\sl\begin{tabbing}}{\end{tabbing}\end{minipage}}
\newcommand{\vem}{\bfseries}
\newcommand{\quotedstring}[1]{{#1}}
\newcommand{\typename}[1]{{#1}}
\newcommand{\literal}[1]{{#1}}
\newcommand{\mi}[1]{\mathit{#1}}

% comments and notes
\newcommand{\comment}[1]{}
%\newcommand{\note}[1]{{\bf $\clubsuit$ #1 $\spadesuit$}}

% figures
\newcommand{\figurebox}[1]
        {\fbox{\begin{minipage}{\textwidth} #1 \medskip\end{minipage}}}
%        {\fbox{\begin{minipage}{\textwidth}\begin{center} #1 \end{center}\medskip\end{minipage}}}
\newcommand{\boxfig}[3]
        {\begin{figure*}\figurebox{#3\caption{\label{fig:#1}#2}}\end{figure*}}
\newcommand{\figref}[1]
        {Figure~\ref{fig:#1}}

% typing rules (not used here)
\newcommand{\ttag}[1]{\mbox{\textsc{\small(#1)}}}
\newcommand{\infer}[3]{\mbox{#1 }\ba{c} #2 \\ \hline #3 \ea}
\newcommand{\irule}[2]{{\renewcommand{\arraystretch}{1.2}\ba{c} #1 
                        \\ \hline #2 \ea}}
\newlength{\trulemargin}
\newlength{\trulewidth}
\newlength{\srulewidth}
\setlength{\trulemargin}{0.80cm}
\setlength{\trulewidth}{40.0mm}
\setlength{\srulewidth}{3.0cm}
\newenvironment{trules}{$\vspace{0.5em}\ba{p{\trulemargin}@{~}p{\trulewidth}@{~}p{\trulemargin}}}{\ea$}
\newenvironment{srules}{$\vspace{0.5em}\ba{p{\trulemargin}@{~}p{\srulewidth}}}{\ea$}
\newcommand{\laxiom}[2]{\ttag{#1} & $ #2 \hfill\ }
\newcommand{\raxiom}[2]{\hfill #2 $& \hfill \ttag{#1}}
\newcommand{\caxiom}[2]{\ttag{#1} & $\hfill #2 \hfill $& \ }
\newcommand{\lrule}[3]{\laxiom{#1}{\irule{#2}{#3}}}
\newcommand{\rrule}[3]{\raxiom{#1}{\irule{#2}{#3}}}
\newcommand{\crule}[3]{\caxiom{#1}{\irule{#2}{#3}}}
\newcommand{\lsrule}[3]{\lsaxiom{#1}{\irule{#2}{#3}}}
\newcommand{\rsrule}[3]{\rsaxiom{#1}{\irule{#2}{#3}}}
\newcommand{\nl}{\end{trules}\\[0.5em] \begin{trules}}
\newcommand{\snl}{\end{srules}\\[0.5em] \begin{srules}}

% commas and semicolons
\newcommand{\comma}{,\,}
\newcommand{\commadots}{\comma \ldots \comma}
\newcommand{\semi}{;\mbox{;};}
\newcommand{\semidots}{\semi \ldots \semi}

% spacing
\newcommand{\gap}{\quad\quad}
\newcommand{\biggap}{\quad\quad\quad}
\newcommand{\nextline}{\\ \\}
\newcommand{\htabwidth}{0.5cm}
\newcommand{\tabwidth}{1cm}
\newcommand{\htab}{\hspace{\htabwidth}}
\newcommand{\tab}{\hspace{\tabwidth}}
\newcommand{\linesep}{\ \hrulefill \ \smallskip}

% math stuff
\newenvironment{myproof}{{\em Proof:}}{$\Box$}
\newenvironment{proofsketch}{{\em Proof Sketch:}}{$\Box$}
\newcommand{\Case}{{\em Case\ }}

% make ; a delimiter in math mode
% \mathcode`\;="8000 % Makes ; active in math mode
% {\catcode`\;=\active \gdef;{\;}}
% \mathchardef\semicolon="003B

% reserved words
\newcommand{\mathem}{\bf}

% brackets
\newcommand{\set}[1]{\{#1\}}
\newcommand{\sbs}[1]{\lquote #1 \rquote}

% arrays
\newcommand{\ba}{\begin{array}}
\newcommand{\ea}{\end{array}}
\newcommand{\bda}{\[\ba}
\newcommand{\eda}{\ea\]}
\newcommand{\ei}{\end{array}}
\newcommand{\bcases}{\left\{\begin{array}{ll}}
\newcommand{\ecases}{\end{array}\right.}

% \cal ids
\renewcommand{\AA}{{\cal A}}
\newcommand{\BB}{{\cal B}}
\newcommand{\CC}{{\cal C}}
\newcommand{\DD}{{\cal D}}
\newcommand{\EE}{{\cal E}}
\newcommand{\FF}{{\cal F}}
\newcommand{\GG}{{\cal G}}
\newcommand{\HH}{{\cal H}}
\newcommand{\II}{{\cal I}}
\newcommand{\JJ}{{\cal J}}
\newcommand{\KK}{{\cal K}}
\newcommand{\LL}{{\cal L}}
\newcommand{\MM}{{\cal M}}
\newcommand{\NN}{{\cal N}}
\newcommand{\OO}{{\cal O}}
\newcommand{\PP}{{\cal P}}
\newcommand{\QQ}{{\cal Q}}
\newcommand{\RR}{{\cal R}}
\newcommand{\TT}{{\cal T}}
\newcommand{\UU}{{\cal U}}
\newcommand{\VV}{{\cal V}}
\newcommand{\WW}{{\cal W}}
\newcommand{\XX}{{\cal X}}
\newcommand{\YY}{{\cal Y}}
\newcommand{\ZZ}{{\cal Z}}

% misc symbols
\newcommand{\dhd}{\!\!\!\!\!\rightarrow}
\newcommand{\Dhd}{\!\!\!\!\!\Rightarrow}
\newcommand{\ts}{\,\vdash\,}
\newcommand{\la}{\langle}
\newcommand{\ra}{\rangle}
\newcommand{\eg}{{\em e.g.}}

% misc identifiers
\newcommand{\dom}{\mbox{\sl dom}}
\newcommand{\fn}{\mbox{\sl fn}}
\newcommand{\bn}{\mbox{\sl bn}}
\newcommand{\sig}{\mbox{\sl sig}}
\newcommand{\IF}{\mbox{\mathem if}}
\newcommand{\OTHERWISE}{\mbox{\mathem otherwise}}
\newcommand{\expand}{\prec}
\newcommand{\weakexpand}{\prec^W}
\newcommand{\spcomma}{~,~}

%\newcommand{\inst}{\mbox{\mathem inst}}
\newcommand{\trans}[1]{\la\!\la#1\ra\!\ra}
\newcommand{\remark}[1]{{\bf $\clubsuit$ #1 $\spadesuit$}}
\newcommand{\todo}[1]{\remark{to do: #1}}
%\newcommand{\J}{\justifies}
%\newcommand{\U}{\using}

% names
\newcommand{\Scala}{\mbox{\textsc{Scala}}}
\newcommand{\Java}{\mbox{\textsc{Java}}}

%\renewcommand\textfraction{.05}
%\renewcommand\floatpagefraction{.9}
%\renewcommand\topfraction{.8}

%%%%%%%%%%%%%%%%%%%%%%%%%%%%%%%%%%%%%%%
%   Language abstraction commands     %
%%%%%%%%%%%%%%%%%%%%%%%%%%%%%%%%%%%%%%%

%% Relations
% Subtype 
\newcommand{\sub}{<:}
% Type assignment
\newcommand{\typ}{:}
% reduction
\newcommand{\reduces}{\;\rightarrow\;}
% well-formedness
\newcommand{\wf}{\;\mbox{\textbf{wf}}}
\newcommand{\nswf}{\mbox{\textbf{wf}}}
\newcommand{\wfe}{\;\mbox{\textbf{wfe}}}
\newcommand{\nswfe}{\mbox{\textbf{wfe}}}

%% Operators
% Type selection
\newcommand{\tsel}{\#}
% Function type
\newcommand{\tfun}{\rightarrow}
\newcommand{\dfun}[3]{(#1\!:\!#2) \Rightarrow #3}
% Conjunction
\newcommand{\tand}{\wedge}
% Disjunction
\newcommand{\tor}{\vee}
% Singleton type suffix
\newcommand{\sing}{.\textbf{type}}

%% Syntax
% Header for typing rules
\newcommand{\judgement}[2]{{\bf #1} \hfill #2}
% Refinement
\newcommand{\refine}[2]{\left\{#1 \Rightarrow #2 \right\}}
\newcommand{\mlrefine}[2]{\{#1 \Rightarrow #2 \}}
% Field definitions
\newcommand{\ldefs}[1]{\left\{#1\right\}}
\newcommand{\mlldefs}[1]{\{#1\}}
% Member sequences
\newcommand{\seq}[1]{\overline{#1}}
% Lambda
\newcommand{\dabs}[3]{(#1\!:\!#2)\Rightarrow #3}
\newcommand{\abs}[3]{\lambda #1\!:\!#2.#3}
% Method Application
\newcommand{\mapp}[3]{#1.#2(#3)}
% Substitution
\newcommand{\subst}[3]{[#1/#2]#3}
% Object creation
\newcommand{\new}[3]{\textbf{val }#1 = \textbf{new }#2 ;\; #3}
\newcommand{\mlnew}[3]{\textbf{val }#1 = \textbf{new }#2 ;\;\\&#3}
%\renewcommand{\new}[3]{#1 \leftarrow #2 \,\textbf{in}\, #3}
% Field declaration
\newcommand{\Ldecl}[3]{#1 : #2..#3}%{#1 \operatorname{>:} #2 \operatorname{<:} #3}
\newcommand{\ldecl}[2]{#1 : #2}
\newcommand{\mdecl}[3]{#1 : #2 \tfun #3}
% Top and Bottom
\newcommand{\Top}{\top}%{\textbf{Top}}
\newcommand{\Bot}{\bot}%\textbf{Bot}}
% Environment extension
%\newcommand{\envplus}[1]{\uplus \{ #1 \}}
\newcommand{\envplus}[1]{, #1}
% Reduction
\newcommand{\reduction}[4]{#1 \operatorname{|} #2 \reduces #3 \operatorname{|} #4}

% Sugar
\newcommand{\arrow}[2]{#1\rightarrow_s#2}
\newcommand{\fun}[4]{\textbf{fun } (#1:#2)\;#3\;#4}
\newcommand{\app}[2]{(\textbf{app }#1\;#2)}
\newcommand{\mlapp}[2]{(\textbf{app }#1\;\\&#2)}
\newcommand{\cast}[2]{(\textbf{cast }#1\;#2)}

\newcommand{\lindent}{\hspace{-4mm}}

% Logical relations
\newcommand{\relv}[4]{\mathcal{V}_{#1;#2;#3}\llbracket#4\rrbracket}
\newcommand{\rele}[4]{\mathcal{E}_{#1;#2;#3}\llbracket#4\rrbracket}
\newcommand{\rels}[3]{\mathcal{\supseteq}_{#1}\llbracket#2;#3\rrbracket}
\newcommand{\relg}[3]{\mathcal{\supseteq^!}_{#1;#2}\llbracket#3\rrbracket}
\newcommand{\irred}[2]{\text{irred }(#1,#2)}
\newcommand{\andl}{\;\wedge\;}
\newcommand{\orl}{\vee}
\newcommand{\impliesl}{\rightarrow}
\newcommand{\reductionl}[5]{#1 \operatorname{|} #2 \;\rightarrow^{#5}\; #3 \operatorname{|} #4}
\newcommand{\ds}{\,\vDash\,}


%\usepackage{beamerthemesplit}
\usecolortheme{seagull}
\useinnertheme{circles}
\useoutertheme{infolines}

\title{DOT: Dependent Object Types}
\subtitle{Semester Project, Spring 2012}
\author{Nada Amin}
\institute{EPFL}
\date{}

\begin{document}

\frame{\titlepage}

\section{Introduction}

\subsection{What is DOT?}

\begin{frame}
\frametitle{DOT: Dependent Object Types}
\begin{itemize}
\item type-theoretic foundation of Scala and languages like it
\item models:
\begin{itemize}
\item path-dependent types
\item abstract type members
\item mixture of nominal and structural typing via refinement types
\end{itemize}
\item does not model:
\begin{itemize}
\item inheritance and mixin composition
\item what's currently in Scala
\end{itemize}
\end{itemize}
\end{frame}

\begin{frame}
\frametitle{DOT Syntax}
\begin{columns}
\begin{column}[t]{5cm}
\begin{block}{term $t$}
\begin{itemize}
\item variable\\$x$
\item lambda abstraction\\$\abs x T t$
\item function application\\$\app t {t'}$
\item field selection\\$t.l$
\item object creation expression\\$\new x {T_c \ldefs{\seq{l = v}}} t$
\end{itemize}
\end{block}
\end{column}
\begin{column}[t]{5cm}
\begin{block}{type $T$}
\begin{itemize}
\item selection\\$p.L$
\item refinement\\$T \refine z {\seq{D}}$
\item function\\$T \tfun T'$
\item intersection\\$T \tand T'$
\item union\\$T \tor T'$
\item $\Top$, $\Bot$
\end{itemize}
\end{block}
\end{column}
\end{columns}
\end{frame}

\begin{frame}
\frametitle{DOT Judgments}
\begin{columns}
\begin{column}[t]{5cm}
\begin{block}{Typing Judgments}
\begin{itemize}
\item type assignment\\$\Gamma \ts t \typ T$
\item subtyping\\$\Gamma \ts S \sub T$
\item well-formedness\\$\Gamma \ts T \wf$
\item membership\\$\Gamma \ts t \ni D$
\item expansion\\$\Gamma \ts T \expand_z \seq{D}$
\end{itemize}
\end{block}
\end{column}
\begin{column}[t]{5cm}
\begin{block}{Small-Step Operational Semantics}
\begin{itemize}
\item reduction\\$\reduction t s {t'} {s'}$
\end{itemize}
\end{block}
\end{column}
\end{columns}
\end{frame}

\subsection{DOT Program Example}

\begin{frame}
\frametitle{Basics}
\framesubtitle{Booleans, Error, \ldots}
\begin{align*}
&\mlnew {\mi{root}} {\Top \mlrefine r {\\
&\ \Ldecl {\mi{Unit}} \Bot \Top\\
&\ \ldecl {\mi{unit}} {\Top \tfun {r.\mi{Unit}}}\\
&\ \Ldecl {\mi{Boolean}} \Bot {\Top \mlrefine z {\\
&\ \gap \ldecl {\mi{ifNat}} {(r.\mi{Unit} \tfun r.\mi{Nat}) \tfun (r.\mi{Unit} \tfun r.\mi{Nat}) \tfun r.\mi{Nat}}\\
&\gap}}\\
&\ \ldecl {\mi{false}} {{r.\mi{Unit}} \tfun {r.\mi{Boolean}}}\\
&\ \ldecl {\mi{true}} {{r.\mi{Unit}} \tfun {r.\mi{Boolean}}}\\
&\ \ldecl {\mi{error}} {{r.\mi{Unit}} \tfun \Bot}\\
&\ \ldots\\
&}{\ldefs{\ldots \seq{(l = v)} \ldots }}}{ \ldots }
\end{align*}
\end{frame}

\begin{frame}
\frametitle{Basics (Continued)}
\framesubtitle{Booleans, Error, \ldots}
\begin{align*}
&\mlldefs{\\
&\ \mi{unit}  = \abs x \Top {\new u {\mi{root}.\mi{Unit}} u}\\
&\ \mi{false} = \abs u {\mi{root}.\mi{Unit}} {\\
&\ \gap \mlnew {\mi{ff}} {\mi{root}.\mi{Boolean} \mlldefs{\\
&\ \gap\gap {\mi{ifNat}} = \abs t {\mi{root}.\mi{Unit} \tfun \mi{root}.\mi{Nat}} {\\&\gap\gap\gap\abs e {\mi{root}.\mi{Unit} \tfun \mi{root}.\mi{Nat}}\\&\gap\gap\gap\app e {\mi{root}.\mi{unit}}}\\
&\ \gap}}{
\ \gap \mi{ff}}
&}\\
&\ \mi{error} = \abs u {\mi{root}.\mi{Unit}} {\app {\mi{root}.\mi{error}} u}\\
&\ \ldots\\
&}
\end{align*}
\end{frame}

\subsection{Contributions}
\begin{frame}
\frametitle{Outline}
\tableofcontents
\end{frame}

\section{Counterexamples}

\subsection{Subtyping Transitivity}

  \begin{frame}
    \frametitle{No Subtyping Transitivity to No Preservation}
\begin{columns}
\begin{column}[t]{6cm}
\begin{enumerate}
\item Start with 3 types $S$, $T$, $U$ st $S \sub T$ and $T \sub U$ but $S \not\sub U$.
\item Create function of type $S \tfun S$.
\item Cast it to $S \tfun T$.
\item Cast it to $S \tfun U$.
\item After some reduction step, the first cast vanishes and we need to cast directly from $S \tfun S$ to $S \tfun U$.
\end{enumerate}
Note: The 3 types don't need to be realizable but must be expressible within a realizable universe.
\end{column}
\begin{column}[t]{4cm}
\begin{block}{Code Recipe}
\begin{align*}
&\mlnew u \ldots {
\ \mlapp{\abs x \Top x}{
\ \ \mlapp{\abs f {S \tfun U} f}{
\ \ \ \mlapp{\abs f {S \tfun T} f}{
\ \ \ \ \mlapp{\abs f {S \tfun S} f}{
\ \ \  \ \ \abs x S x}}}}}
\end{align*}
\end{block}
\end{column}
\end{columns}
  \end{frame}

  \begin{frame}
    \frametitle{Non-Expanding Types and Subtyping Transitivity}
\begin{align*}
\Top & \mlrefine u {\\
&\ \Ldecl {\mathit{Bad}} {\Bot} {u.\mathit{Bad}}\\
&\ \Ldecl {\mathit{Good}} {\Top \refine z {\Ldecl L \Bot \Top}} {\Top \refine z {\Ldecl L \Bot \Top}}\\
&\ \Ldecl {\mathit{Lower}} {u.\mathit{Bad} \tand u.\mathit{Good}} {u.\mathit{Good}}\\
&\ \Ldecl {\mathit{Upper}} {u.\mathit{Good}} {u.\mathit{Bad} \tor u.\mathit{Good}}\\
&\ \Ldecl X {u.\mathit{Lower}} {u.\mathit{Upper}}\\
}&
\end{align*}

\begin{align*}
S &= u.\mathit{Bad} \tand u.\mathit{Good}\\
T &= u.\mathit{Lower}\\
U &= u.X \refine z {\Ldecl L \Bot \Top}
\end{align*}
  \end{frame}

\subsection{Narrowing}

  \begin{frame}
    \frametitle{Functions as Objects}
\begin{align*}
&\mlnew u {\Top \refine z {\Ldecl C {\Top \tfun \Top} {\Top \tfun \Top}} \ldefs{}} {
\mlnew f {u.C \ldefs{}} {
\ldots
}}
\end{align*}
  \end{frame}

\subsection{Path Equality}

  \begin{frame}
    \frametitle{Path Equality}
  \end{frame}

\section{Patches}

  \begin{frame}
    \frametitle{Patches}
  \end{frame}

\section{Conclusion}

  \begin{frame}
    \frametitle{Conclusion}
  \end{frame}

\end{document}
